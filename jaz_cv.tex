\textbf{IDALIA JAZMÍN MACÍAS CUÉLLAR}

+52 477 703 6136 \textbar{}
\href{mailto:psic.jazminmc@gmail.com}{\nolinkurl{psic.jazminmc@gmail.com}}

Cédulas profesionales: 13566051,

\textbf{RESUMEN PROFESIONAL}

Psicoterapeuta clínica con experiencia en el tratamiento de trastornos
emocionales y mentales e historial de trabajo con niños, adolescentes y
adultos, abordando desafíos académicos, sociales, de salud mental,
trauma y dificultades en las relaciones interpersonales, para lograr
fortalezas y éxitos individuales. Especializada en enfoques terapéuticos
integrativos, que incluyen la terapia Sistémica relacional, terapia
Centrada en la persona y terapia con enfoque Psicodinámico, además de la
terapia breve. Competente en la identificación de síntomas de trastornos
de la mente y personalidad, por medio de la evaluación del
comportamiento y el estado mental del consultante, a través de la
observación, la entrevista y la administración de encuestas.
Comprometida en la capacitación y actualización profesional constante.

\textbf{EDUCACIÓN}

\textbf{Universidad Iberoamericana Agos 2022 -- Jul 2024}

Posgrado en Psicoterapia Clínica León, Gto., Méx.

\textbf{Universidad del Valle de Atemajac Sept 2017 -- Dic 2020}

Licenciatura en Psicología León, Gto., Méx.

\textbf{EXPERIENCIA}

\textbf{AR Programas de salud mental}

\emph{Dooney \& Bourke \textbar{} Jul 2023 -- Actual}

\begin{itemize}
\item
  Terapia breve a los trabajadores de las diversas áreas de la empresa
\item
  Contención y canalización de casos según las necesidades particulares
\end{itemize}

\emph{Manufacturas DC \textbar{} Abr 2023 -- Actual}

\begin{itemize}
\item
  Terapia breve a familiares y trabajadores de la empresa
\end{itemize}

\textbf{Psicoterapia clínica privada}

\emph{Atención online y presencial \textbar{} Dic 2020 -- Actual}

\textbf{Universidad de Guanajuato}

\emph{Chat Psicológico UG} \textbf{\textbar{}} \emph{Feb 2022} --
\emph{Jul 2022}

\begin{itemize}
\item
  Atención y orientación psicológica por medio de plataforma virtual y
  videollamada
\item
  Conocimiento de protocolos de primeros auxilios psicológicos y GAUS.
\end{itemize}

\emph{Escuela de Nivel Medio Superior León} \textbar{} \emph{Ene 2021 --
Ene 2022}

\begin{itemize}
\item
  Servicio social como orientadora psicológica del Sistema Integral de
  Salud Estudiantil.
\item
  Facilitación de talleres para la promoción de la salud emocional y
  prevención del suicidio.
\end{itemize}

\textbf{Prácticas profesionales}

\emph{Centro Educativo de Servicios para la Comunidad (CESCOM)
\textbar{} Agos 2022 -- Jul 2024}

\emph{Centro de apoyo a la comunidad (CEAC) \textbar{} Sept 2019 -- Dic
2020}

\textbf{HABILIDADES Y APTITUDES}

\begin{itemize}
\item
  Diagnóstico diferencial con un marco referencial del DSM-V
\item
  Aplicación, revisión e interpretación de pruebas psicométricas y test
  proyectivos.
\item
  Primeros auxilios psicológicos
\item
  Intervención en crisis psicológica y suicida
\item
  Impartición de talleres para la prevención del suicidio, salud mental,
  sentido de vida y orientación vocacional.
\end{itemize}
